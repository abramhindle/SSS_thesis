% Allow relative paths in included subfiles that are compiled separately
% See https://tex.stackexchange.com/questions/153312/
\providecommand{\main}{..}

\documentclass[\main/thesis.tex]{subfiles}
\externaldocument{}



\begin{document}
\chapter{Datasets}
\label{sec:datasets}
\section{Why is Data Needed?}
This project requires a \textit{virtual synthesizer} capable of producing sounds with a variety of characteristics (as long as a fraction of these sounds can be suitable replacements for percussion). On the other-hand, \textit{virtual ear} has a more concrete task: the separation of drum-like sounds from other synthesizer outputs. Rather than manually defining the characteristics which distinguish percussive sounds from all other types, we train the virtual ear by example. To this end, we gathered 3 databases of percussive sounds. The upcoming Chapters~\ref{chap:represent_sound} and~\ref{implementation} will cover how these datasets are transformed and used for training.


% \section{Details of Dataset}

\section{Download Instructions}
FreeDB, survey data, and randomly generated sounds can be downloaded from: \url{https://zenodo.org/record/3994999}\\
RadarDB requires lengthy downloads and processing. The script for its automated creation can be found under the \enquote{getting\_data} directory of our project code: \footnote{\url{https://github.com/imilas/Synths_Stacks_Search}}
\section{Details}
MixedDB is a large set of 2 second or shorter drum samples aggregated from personal libraries. FreeDB is a dataset of free drum kits from the sample-swap project \footnote{https://sampleswap.org/} which we organized into 5 groups. RadarDB is a set of drum sounds aggregated from royalty free sources such as music radar \footnote{https://www.musicradar.com/}. We put together 3 databases of drums using these sources. We also created a database of synthetic noise from 1, 3, and 5 stacked virtual synthesizers. We have made our dataset of free-drum sounds available for download. The scripts used to download and process royalty free samples are also made available. 

We prioritize making generalizable tools which can learn from and produce a variety of different sounds. We utilize these databases depending on the task at hand. At times, we merged or purged drum groups to simplify tasks.

\begin{table}[]
% \centering
% \caption{Curated databases, including NoiseDB}
\begin{tabular}{ |c|c| } 
\hline
DB Name & Categories \\ \hline
FreeDB  & Kicks:533 - Snares:372 - Claps:230 - Hats:105 - Other:281            \\ \hline
RadarDB & \makecell{Kicks:1054 - Snares:842 - Claps:353 \\ Toms:349 - Hats:1561 - Rims:131 - Shakers:121} \\ \hline
MixedDB & Kicks:533 - Snares:372 - Claps:230 - Hats:105 - Others:281                     \\ \hline
NoiseDB & 1 Stack:2000 - 3 Stacks:2000 - 5 Stacks:2000                         \\ \hline
\end{tabular}
    % \caption{Overview of our datasets.}
    \label{table:all_db}
\end{table}
% \begin{table}[h]
% \centering
%  \resizebox{\linewidth}{!}{\begin{tabular}{|l|l|l|l|l|l|l|l|l|l|}
% \hline
% DB Name & kick & snare & clap & tom\_high & tom\_mid & tom\_low & hihat\_closed &  hihat\_open & rim \\ \hline
% MixedDB & 648 & 732 & 118 & 179 & 139 &  188 & 187 & 280 & 105 \\\hline
% \end{tabular}}
% \caption{Database 1: Mixed sources}
% \label{db:self}
% \end{table}

% \begin{table}[h]
% \centering
% \begin{tabular}{|l|l|l|l|l|l|l|l|l|}
% \hline
% DB Name & kick & snare & clap & tom & clap & hat & rim & shaker  \\ \hline
% RadarDB & 1054 & 842   & 353 & 349 &  353 & 1561& 131 & 121 \\ \hline
% \end{tabular}
% \caption{Database 2: Royalty free sounds sourced from \enquote{Music Radar}}
% \label{db:radar}
% \end{table}

% \begin{table}[h]
% \centering
% \begin{tabular}{|l|l|l|l|l|l|}
% \hline
%  DB Name & kick & snare & clap & hat & other \\\hline
%  FreeDB & 533 & 372 & 230 & 105 & 281 \\ \hline
% \end{tabular}
% \caption{Database 3: Free sounds sourced from the \enquote{Sample Swap} project. Simplified for our purposes. The version available for download contains more sample groups. The \enquote{other} category contains a variety of percussive sounds.}
% \label{db:sampleswap}
% \end{table}


% \begin{table}[h]
% \centering
% \begin{tabular}{|l|l|l|l|}
% \hline
%  Synth Noise Type & 1 Stack & 3 Stacks  & 5 Stacks \\ \hline
%  Number of Examples & 2000 & 2000 & 2000 \\ \hline
% \end{tabular}
% \caption{Database of random noise examples from our virtual synthesizers}
% \label{db:noise}
% \end{table}


\end{document}