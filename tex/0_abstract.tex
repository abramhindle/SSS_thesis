% Allow relative paths in included subfiles that are compiled separately
% See https://tex.stackexchange.com/questions/153312/
\providecommand{\main}{..}
\documentclass[\main/thesis.tex]{subfiles}

\begin{document}

\begin{abstract} 
We create systems that automatically generate and categorize drum synthesizer programs, and measure the success of these systems by conducting manual listening tests of the generated sounds. Can we create systems of drum sound generation where the majority of outputs sound like drums to human listeners?

Recent advancements in digital synthesis, heuristic search, and neural networks can be utilized for sound generation; yet the lack of access to large audio datasets, the problem of open set recognition, and high computational costs persist as barriers towards the expansion of sound libraries using these techniques.
We present a number of approaches taken towards the automatic generation of synthesizer programs which mimic one-shot percussive sounds. 
We focus on the implementation an automatic system for generation of quick and scalable percussion synthesizers using classical signal processing. Our system relies on virtual ears to automatically listen, find and classify synthesizer programs that mimic percussive sounds.
We demonstrate promising results in both detection and categorization of percussive sounds by representation of digital audio through Fourier transformations and autoencoder embeddings. Manual listening tests of the generated sounds indicate that the system can successfully generate drum synthesizers and categorize drum sounds.

Furthermore, we share our curated datasets of free percussive sounds. These datasets can be used for the replication of our work, or to facilitate new research. 
\end{abstract}


% \begin{abstract} CSMC Abstract\\
% Can we generate drum
% We present an approach for the automatic generation of synthesizer programs for one-shot percussive sounds. 
% Recent advancements in digital synthesis, heuristic search, and neural networks can be utilized for sound generation. 
% Yet the need for data, the problem of open set recognition, and high computational costs persist as barriers towards the expansion of sound libraries using these techniques. 
% We generate quick, scalable, percussion synthesizers using classical signal processing. 
% We train drum classifiers to find and classify synthesizer programs that mimic percussive sounds. 
% We use features from Fourier transformations and autoencoder embeddings to train machine learning classifiers.
% Manual listening tests of the generated sounds demonstrates the system can successfully generate drum synthesizers and categorize drum sounds.
% To facilitate future research, we share our curated dataset of free percussive sounds.

% \end{abstract}


\end{document}