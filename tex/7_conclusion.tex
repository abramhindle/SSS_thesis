
\providecommand{\main}{..}
\documentclass[\main/thesis.tex]{subfiles}
\externaldocument{}



\begin{document}

\chapter{Conclusion, Validity, and Future Work} 

\section{Conclusion}  We built a generative system for creation of percussive sounds via automatic programming of virtual synthesizers. We verified its results with human listeners. In the surveys conducted, the majority of sounds generated by the system were percussive and categorized accurately. While many avenues of improvement are available, the implementation outlined here satisfies our stated goal. 
Our work enables not only the creation of new libraries of percussion sounds, but new synthesizer programs which can be modified and studied. 
Manual listening tests revealed much room for improvement, particularly with accurate separation of percussive sounds from the infinitely large set of non-percussive sounds. We had some success in our utilization of latent representations of autoencoder networks as low-dimensional representations of short sound files.
Manual listening tests revealed much room for improvement, particularly with accurate separation of percussive sounds from the infinitely large set of non-percussive sounds. We had some success in our utilization of latent representations of autoencoder networks as low-dimensional representations of short sound files. \\ 
\section{Threats to Validity}
\subsection{Internal Validity}
 The lack of consistency in training and accuracy measurements makes comparisons between TPEs and MEMs difficult. The models have been trained on different subsets of the 3 different drum datasets, and cross validation was utilized in measuring the MEMs but not the TPEs. The design of the virtual synthesizer---its parameters and parameter values---is loosely based on commonly found parameters in VST synthesizers, therefore, we cannot guarantee that these results will translate to other virtual synthesizers. We also did not measure what percentage of the randomly generated sounds are percussive before being filtered by the virtual ear. 
\subsection{External Validity}
 We did not measure, nor can we guarantee, the novelty of the generated drum sounds. In addition, since both survey subjects have a preference for \enquote{experimental} electronic music, which may influence their perception of what is and is not an acceptable drum sound. Furthermore, both survey takers played a role in the implementation of the project, which may influence their leniency in output sounds as percussive or non-percussive. 
 
 
\section{Future Work} Effective implementation of few-shot learning is a priority, as it will allow for a larger variety of sounds to be generated using our methods and possibly address the OSR problem. Program generation can be improved by reinforcement learning and other heuristics, however, we need to ensure that there is a stronger agreement between the synthetic ear's scores and human listeners. We also more specialized autoencoder architectures and training methods can improve feature extraction.

\end{document}