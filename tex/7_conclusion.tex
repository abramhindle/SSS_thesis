
\providecommand{\main}{..}
\documentclass[\main/thesis.tex]{subfiles}
\externaldocument{}



\begin{document}

\chapter{Conclusion, Validity, and Future Work} 

\section{Conclusion}  We built a generative system for creation of percussive sounds via automatic programming of virtual synthesizers. We verified its results with human listeners.
Our work enables not only the creation of new libraries of percussion sounds, but new synthesizer programs which can be modified and studied. 
Manual listening tests revealed much room for improvement, particularly with accurate separation of percussive sounds from the infinitely large set of non-percussive sounds. We had some success in our utilization of latent representations of autoencoder networks as low-dimensional representations of short sound files.
Manual listening tests revealed much room for improvement, particularly with accurate separation of percussive sounds from the infinitely large set of non-percussive sounds. We had some success in our utilization of latent representations of autoencoder networks as low-dimensional representations of short sound files. \\ 
\section{Threats to Validity}  The lack of consistency in training and accuracy measurements makes comparisons between TPEs and MEMs difficult. Many arbitrary design decisions have been made, particularly in the design of our TPE classifiers and selection of datasets for training and testing. We cannot quantify the novelty of our results. Our requirement for training data remains high \\
\section{Future Work} Effective implementation of few-shot learning is a priority. Program generation may be improved by reinforcement learning and other heuristics.



\end{document}