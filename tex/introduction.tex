% Allow relative paths in included subfiles that are compiled separately
% See https://tex.stackexchange.com/questions/153312/
\providecommand{\main}{..}
\documentclass[\main/thesis.tex]{subfiles}

\begin{document}

\chapter{Introduction}
The rise of Digital Audio Workstations (DAW) \cite{leider2004digital} and Virtual Studio Technology (VST) based plug-ins \cite{tanev2013virtual} have rapidly transformed the sonic and material landscape of music production in the recent years. Coupled with this rise in popularity is a vast array of commercial products and services dedicated to satiating the need of amateur and professional music producers for unique sounds; most commonly via audio samples: one-shot drum samples, long sustained notes (commonly referred to as pads or textures), and loops (percussive or melodic) are common deliverables. Two notable examples of these commercial services are \textit{loopmasters}\footnote{loopmasters.com} and \textit{splice.com}\footnote{splice.com}. Furthermore, VST plug-ins can emulate complex audio synthesizers and effects which some producers may find daunting or time consuming to work with from scratch. In many cases VST plug-in vendors or unaffiliated enthusiasts sell additional presets for these plugins, targeted towards producers who do not have the time or interest in creating their own. The flexibility of the VST technology allows producers to further modify these presets until their desired sound is reached.

We want to leverage AI technologies, such as heuristic search and random search, to produce new drum samples, by finding synthesizers and their appropriate configurations that can produce various kinds of drum samples. Effectively we will combine machine listening and code generation of synthesizers to find drum samples.

Our work is motivated by the idea of finding new, convenient methods for the expansion of a music producer's library of sounds. Primarily with generation of novel, one-shot audio samples but also with automated search and creation of presets for virtual synthesizers. Using the generation of short, percussive audio samples as a starting point, this project is a proof of concept for a promising avenue towards our motivational goal.


\section{Methodology}
Our attempt is centered around a generative pipeline of audio. We want random sounds to be created by a tractable source, we also want to evaluate these sounds so we can guide the random sounds generator towards desired sounds. Instead of learning weights and parameters in an audio-generating neural network, we wish to generate, search, and tune synthesizers to produce percussion sounds. Following this idea, we found the proper implementation of 2 major components to be crucial:

\begin{itemize}
    \item \textit{Virtual Synthesizer}: A flexible, deter\-min\-istic, and tract\-able gener\-ator which can create audio. 
    % The core of our implementation made use of pippi, \footnote{https://github.com/luvsound/pippi} a fast, offline focused python DSP library with a C backend. We additionally used the scipy\cite{jones2001scipy} library a few audio effects that we found lacking in pippi. 
    \item \textit{Virtual Ear}: An ear that returns an evaluation of an audio sample; estimating the effectiveness of an audio sample's fulfillment of a producers requirements. The ear's evaluation guides the generation process towards a desired path, making it a crucial component of our pipeline. 
\end{itemize}
% AH: Optional. Can be removed or reduced later
Our components are designed with modularity and parallelizability in mind. This allows each component to be debugged, modified, and improved without requiring modifications in other components while additionally increasing the scalability and speed of experiments. 
Section \ref{impl} contains further discussion of the components as well as the code that glues the project together.

While the main focus of this project is the generation of novel percussive sounds, our methodology indicates promising results with regards to creation of new presets for any virtual synth without the need for a-priori knowledge of the functions or its parameters (i.e the effect of parameter modulation on the sonic output). We also demonstrate the viability of a virtual synthesizers based on Digital Signal Processing (DSP) methods for fast, unsupervised creation of novel audio, discussed in more detail in section \ref{related}. 

\section{Related Work and Contradistinctions}
\label{related}
Numerous deep, neural network models have been proposed and utilized for the purpose of signal generation in recent years. WaveGans and WaveNet have been subject to significant improvements and experiments since their proposal \cite{nsynth2017,yamamoto2019parallel,oord2017parallel}. Even more recently Variational AutoEncoders (VAE's) have been utilized for generation of short percussive samples \cite{aouameur2019neural,ramires2019timbfeat}. In this work however, we opt to use digital signal processing methods to create a virtual synthesizer for the generation of audio signals as it provides several unique advantages:
\begin{enumerate}[label=\roman*]
  \item Fast, offline rendering of audio with no reliance on GPU: Currently not possible with state of the art models such as parallel WaveGan \cite{yamamoto2019parallel} and parallel WaveNet \cite{oord2017parallel}. 
  \item Rendering at high sampling rates: Performance speed being a common issue, the standard sampling rate in most audio generation work utilizing neural networks appears to be under 24 khz \cite{yamamoto2019parallel,oord2017parallel,aouameur2019neural,ramires2019timbfeat}. However, a significant number of untrained human ears can detect a change in quality of audio between sampling rates of 192 khz and the industry standard of 44.1 khz \cite{reiss2016meta} with a dramatic increase in quality detection after training. Therefore we can safely assume that most producers would prefer their audio samples to have sampling rates of 44.1 khz or higher. In this work, we fix our sampling rate to the 48 khz standard. 
  \item Neural networks are often viewed as unexplainable black box solutions. Some models such as VAE's can learn an underlying latent space of parameters and capture the "essence" of the different labels in a dataset. However, these spaces are learned in an unsupervised manner and must be manually analysed, perhaps extensively, before they can be understood \cite{esling2018generative}. The use of a virtual synthesizer for audio generation makes our parameters readily understandable and easily modifiable. \\
\end{enumerate}

Automatic programming of virtual synthesizers has also been a topic of interest. In early 2000s, Interactive Genetic Algorithms (IGA's) were utilized for the generation of new sounds with various sound-engines \cite{johnson1999exploring,dahlstedt2001creating}. More recent work by Yee-King et al. \cite{yee2018automatic} used Long short-Term Memory (LSTM) models and genetic algorithms to find the exact parameters used to create a group of sounds. The sounds approximated were made by the same virtual synthesizer, not an external source; making the eventual replication certain even with random search. Since this work appears more focused on pads and textures rather than drums, feature matching appears to not be concerned with the envelope of the sounds but rather the frequency content within arbitrary time windows. Yet another recent, impressive work by Esling et al. used a large dataset of over 10,000 presets for a commercial VST synthesizer to learn a latent parameter space which can be sampled for creation of new audio \cite{esling2019universal}. As stated before, our work explores the rapid approximation of percussion sounds with no previous knowledge about the sonic capabilities of our virtual synthesizer, exploring the actual parameter space rather than its latent representation. 


\section{Data And Project Replication}
\label{data}
Our data is a large set of drum samples aggregated from personal libraries, free drum kits from the sample-swap project \footnote{https://sampleswap.org/} which we further processed to suit our categories, and a large set of drum sounds aggregated from royalty free sources such as musicradar \footnote{https://www.musicradar.com/}. We will make our dataset of free-drum sounds available for download. The scripts used to download and process royalty free samples will also be made available. Further information about downloading our dataset can be found on this project's github page. Current URLs are redacted for double-blind purposes.

Our drum categories are claps, hats, kicks, rims/other, shakers, snares, mid-toms and high-toms. Other categories are chopped guitars, chopped pianos and n-stack-synths (random noise generated by the virtual synth with stack size of n, see \ref{vs}), utilized for learning percussive vs non-percussive sounds. Stack sizes refers to how many synth functions are connected together in a synthesizer.
Some notes about our dataset:


\begin{itemize}
\item Of the 6000 drum-sounds utilized in our work, the kick, snare and hat categories have the largest share at around 20\% each, while the shaker and rim (other) categories have the smallest at 5\% combined. Due to this we only focused on learning from kicks, snares, toms, claps and hats for Phase 1 of training (along with non-percussive groups of sound) \ref{sec:ear}.
\item For Phase 2 of training we only focused on categorizing snares, claps, kicks, hats and other (percussive sounds such as shakers, rims and unusual percussions that we couldn't categorize were grouped into this category). Non-percussive datasets were not used for this phase. 
\item In order to offset bias from data imbalance during training of our models, the categorical cross entropy loss was weighted by the group sizes. 
\item For any given model, 80\% of our data is used for training and 20\% is used for testing. 

\item We limit the size of the n-stack-synths category to 50\% of the total size of our drum dataset. This is done in order to measure whether the features extracted can address the "Open Set Recognition" problem, which will be discussed further in Section \ref{sec:ear}.
\end{itemize}

\end{document}